\chapter*{Introduction}
\phantomsection
\addcontentsline{toc}{chapter}{Introduction}

%\section*{Introduction}
The Internet of Things (IoT) has revolutionized various industries by enabling the interconnection of billions of devices. One of the key technologies for IoT connectivity is Narrowband Internet of Things (NB-IoT). NB-IoT provides low-power, wide-area (LPWA) coverage, making it suitable for applications that require long battery life and extended range.

The main goal of this diploma thesis is to develop a simulation tool for the capacity planning of NB-IoT technology in Network Simulator 3 (NS-3). The developed tool will be based on a Lena-NB module designed to simulate NB-IoT network performance. By utilizing the developed tool, it will be possible to reveal the maximum capacity of a single base station, convergence time, or transmission delay.

In addition, the toolbox will provide a simple user interface for simulation input parameters definition, such as the number of users, message size, and ECL classes. It will also offer results visualization capabilities. The simulation tool will support both UDP and TCP for message transmission, catering to the requirements of permanently connected smart metering devices.

The theoretical part of this thesis will provide a general overview of the available LPWA technologies, a detailed description of NB-IoT technology in all releases, and a discussion on the industrial application of NB-IoT in smart metering systems.

This thesis aims to contribute to the understanding and optimization of NB-IoT technology, enabling efficient capacity planning and performance evaluation for IoT applications.

\section*{Thesis Structure}
This semestral thesis is organized as follows:

\begin{itemize}
    \item Chapter 1 provides an overview of the IoT technology, mainly LPWAN technologies.
    \item Chapter 2 describes NB-IoT in details and its application to metering system.
    \item Chapter 3 presents the simulation scenarios and their results.
    \item Chapter 4 discusses the findings and provides recommendations for future work.
    \item Chapter 5 concludes the thesis.
\end{itemize}